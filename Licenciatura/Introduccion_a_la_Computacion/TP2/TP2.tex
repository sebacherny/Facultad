\documentclass[a4paper]{article}
\usepackage[english,spanish]{babel}
\usepackage[utf8]{inputenc}
\usepackage[T1]{fontenc}
\usepackage{verse}
\usepackage[noend]{algpseudocode}
\usepackage{listings}
%% Sets page size and margins
\usepackage[a4paper,top=2cm,bottom=2cm,left=2cm,right=2cm,marginparwidth=1.75cm]{geometry}
%% Useful packages
\usepackage{parskip}
\usepackage{amsfonts}
\usepackage{amsmath}
\usepackage{graphicx}
\usepackage[usenames]{color}
\usepackage[colorinlistoftodos]{todonotes}
\usepackage[colorlinks=true, allcolors=blue]{hyperref}

\newcommand\ZZ{\mathbb Z}
\newcommand\BB{\mathbb B}

\newcommand{\verso}[1] {
\settowidth{\versewidth}{123456789012345678901234567890}%
\begin{minipage}[t]{\dimexpr\versewidth+1pt\relax}
\begin{verse}[\versewidth]
{\fontfamily{qzc}\selectfont\large
  #1
}
\end{verse}
\end{minipage}\bigskip
}

\title{Trabajo Práctico 2}
\author{Daniela Alban, Sebastian Cherny, Gabriel Krimker y Agustin  Vera.}
\begin{document}
\maketitle

Construímos un léxico usando una estructura de árbol. Consideramos las palabras caminos que van desde el nodo raíz hasta los nodos que tienen un True en $finDePalabra$, donde las flechas serán las letras.


\section*{Clase Nodo}
Le definimos un booleano $finDePalabra$ para saber si termina un camino en ese nodo y un diccionario $referencia$ el cual contiene a las flechas que apuntan a otro nodo.  

La estructura de representación del nodo es $$< finDePalabra:\BB, \,\, referencia:diccionario <letra, Nodo>>.$$

El invariante de representación del TAD Nodo está dado por la condición de que cada letra que esté en el diccionario haga referencia a un nodo.

La construcción de un nuevo Nodo:

$ n \leftarrow Nodo(x) $

El acceso a los campos del nodo $n$:

$n.finDePalabra=x$ (donde $x$ es booleano)

$n.referencia=y$ (donde $y$ es un diccionario que relaciona cada letra con un nodo)


\section*{Clase Lexico}
La estructura de representación es
$$<raiz:ref(Nodo), \,\, cantidadDePalabras: \ZZ>$$

El invariante de representación del TAD LEXICO está dado por las siguientes condiciones: en primer lugar $cantidadDePalabras$ coincide con la cantidad de nodos que tienen True en $finDePalabra$; por otro lado, debe tener forma arbórea (es decir, no puede haber ciclos y cada nodo recibe una única flecha) donde cada flecha se relaciona con un nodo; y finalmente, $raiz$ debe apuntar al primer nodo del árbol (es decir a la raíz).

\subsection*{Los algoritmos de las operaciones del TAD Léxico:}

\subsubsection*{Crear un léxico:}

El constructor del léxico va a tener una referencia al primer nodo del árbol y otra a la cantidad de palabras del léxico.

\textcolor{blue}{NuevoLexico() $\rightarrow$ Lexico:}

$L \leftarrow NuevoLexico ()$

$L.cantidadDePalabras=k$ (donde $k$ es un número entero)

$L.raiz=n$ (donde $n$ es un nodo)

Como crear un léxico es asignarle un valor a $l.cantidadDePalabras $ y un nodo a $l.raiz $, tiene orden \textcolor{red}{O(1)}.

\subsubsection*{Cantidad de palabras:}

\textcolor{blue}{CantidadDePalabras(L) $\rightarrow \ZZ$ :}

$ RV \leftarrow L.cantidadDePalabras $

Como Lexico tiene guardada la cantidad de palabras en $L.cantidadDePalabras$, la función $CantidadDePalabras$ devuelve ese valor en \textcolor{red}{O(1)}.  

\subsubsection*{Existe:}

\textcolor{blue}{Existe(L,P) $\rightarrow \BB $:}

$ letra \leftarrow L.raiz $\hspace{0.5cm} \textcolor{red}{O(1)}

$ i \leftarrow 0	$\hspace{0.5cm} \textcolor{red}{O(1)}

 while ($ i < longitud(P)$ $\wedge$ la letra i-ésima de la palabra $P$ esté relacionada con un Nodo en

$ letra.referencia$) $\{$\hspace{0.5cm} \textcolor{red}{O(1)} \hspace{0.5cm} \textcolor{red}{while: O(longitud(P)) iteraciones} 

$\hspace{0.5cm}$cambiar $letra$ por el nodo que está relacionado con la i-ésima letra de $P$ en $letra.referencia$ \hspace{0.5cm} \textcolor{red}{O(1)}

$\hspace{0.5cm } i \leftarrow i+1 $\hspace{0.5cm} \textcolor{red}{O(1)}

$\}$

$ RV \leftarrow (i = longitud(P) \wedge letra.finDePalabra) $ \hspace{0.5cm} \textcolor{red}{O(1)}
	
   
\subsubsection*{Agregar Palabra:}
\textcolor{blue}{AgregarPalabra (L,P): }

If not $(L.Existe(P)) \{$ \hspace{0.5cm} \textcolor{red}{O(longitud(P))} 

\hspace{0.5cm} if $(longitud(P) \neq 0) \{$\hspace{0.5cm} \textcolor{red}{O(1)}

\hspace{1cm} $i\leftarrow 0$ \hspace{0.5cm} \textcolor{red}{O(1)}

\hspace{1cm} $letra \leftarrow L.raiz$ \hspace{0.5cm} \textcolor{red}{O(1)}

\hspace{1cm} while$(i<longitud(P))\{$ \hspace{0.5cm} \textcolor{red}{O(1)} \hspace{0.5cm} \textcolor{red}{while: O(longitud(P)) iteraciones} 

\hspace{1.5cm} if (la letra i-ésima de $P$ no está relacionada con un Nodo en $letra.referencia$)$\{$\hspace{0.5cm} \textcolor{red}{O(1)}

\hspace{2cm} crear $Nodo (False)$ y relacionar con la i-ésima letra de $P$ en $letra.referencia$\hspace{0.5cm} \textcolor{red}{O(1)} 

\hspace{1.5cm}$\}$
                        
\hspace{1.5cm}	cambiar $letra$ por el nodo relacionado con la i-ésima letra de $P$ en $letra.referencia$\hspace{0.5cm} \textcolor{red}{O(1)}

\hspace{1.5cm} $i \leftarrow i+1$\hspace{0.5cm} \textcolor{red}{O(1)}

\hspace{1cm}$\}$

\hspace{1cm}  $letra.finDePalabra \leftarrow$ True \hspace{0.5cm} \textcolor{red}{O(1)}
  
\hspace{0.5cm} $else$

\hspace{1cm} $L.raiz.finDePalabra \leftarrow$ True \hspace{0.5cm} \textcolor{red}{O(1)}  
                
\hspace{1cm} $L.cantidadDePalabras \leftarrow L.cantidadDePalabras + 1$\hspace{0.5cm} \textcolor{red}{O(1)} 

\hspace{0.5cm}$\}$


\subsubsection*{Eliminar Palabra:}
\textcolor{blue}{L.EliminarPalabra(P):}

If $(Existe(P)) \{$\hspace{0.5cm} \textcolor{red}{O(longitud(p))} 

\hspace{0.5cm}	$letra \leftarrow L.raiz$\hspace{0.5cm} \textcolor{red}{O(1)} 

\hspace{0.5cm} $	i \leftarrow 0$ \hspace{0.5cm} \textcolor{red}{O(1)} 

\hspace{0.5cm} 	while ($i < longitud(P)$)$\{$\hspace{0.5cm} \hspace{0.5cm} \textcolor{red}{O(1)}  \hspace{0.5cm} \textcolor{red}{while: O(longitud(P)) iteraciones}

\hspace{1cm} cambiar $letra$ por el nodo relacionado con la i-ésima letra de $P$ en $letra.referencia$\hspace{0.5cm} \textcolor{red}{O(1)} 
  

\hspace{1cm} $i \leftarrow i+1$	\hspace{0.5cm} \textcolor{red}{O(1)} 

\hspace{0.5cm}$\}$

\hspace{0.5cm} letra.finDePalabra $\leftarrow$ False\hspace{0.5cm} \textcolor{red}{O(1)} 

\hspace{0.5cm} $ L.cantidadDePalabras \leftarrow L.cantidadDePalabras - 1$ \hspace{0.5cm} \textcolor{red}{O(1)} 

$\}$

\end{document}